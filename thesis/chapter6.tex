\فصل{ارزیابی}

\پاراگراف{}
در نهایت به منظور ارزیابی پروژه از داده‌های ترافیکی واقعی \متن‌لاتین{PeMSD7} استفاده می‌کنیم که توسط اداره‌ی حمل و نقل ایالت کالیفرنیا\پانویس{https://pems.dot.ca.gov/} جمع آوری شده است.
این داده مربوط به ۲۲۸ ایستگاه می‌باشد که به صورت تصادفی از میان ۳۹ هزار ایستگاه‌ انتخاب شده‌اند و شامل ۱۲ هزار سطر است که نشان دهنده‌ی سرعت ترافیک در این ۲۲۸ ایستگاه با توالی زمانی پنج دقیقه می‌باشد.
داده‌ها در ماه‌های می تا جون سال ۲۰۱۲ جمع آوری شده‌اند و جهت پاک کردن داده‌های ترافیکی بی‌قاعده روزهای غیرکاری حذف شده‌اند.
برای مقایسه‌ی عملکرد این روش با روش‌های دیگر از معیارهای میانگین مطلق خطا\پانویس{MAE}، میانگین مطلق درصد خطا\پانویس{MAPE} و جذر میانگین مربعات خطا\پانویس{RMSE} استفاده می‌کنیم.
این روش را با روش‌های پایه‌ی شبکه‌ی عصبی پیش‌خور\پانویس{Feed-Forward Neural Network}، مدل خودهمبسته میانگین متحرک و حافظه‌ی کوتاه مدت ماندگار کاملا متصل\پانویس{Full-Connected LSTM} \مرجع{1409.3215} مقایسه خواهیم کرد.

مجموعه داده‌ی \متن‌لاتین{PeMSD7} که در تارنمای \تارنما{https://github.com/VeritasYin/STGCN_IJCAI-18} منتشر شده است. داده‌ای تمیز شده است و آماده‌ی دادن به مدل است. اجرای آموزش با کلاستر لینوکس(پردازنده: \متن‌لاتین{Intel(R) C}، کارت گرافیک: \متن‌لاتین{NVIDIA Ge}) انجام شده است)
تعداد ایپاک ها برابر ۵۰ قرار داده شد و در کل حدود ۳۸ ساعت زمان برد.
در حین آموزش مدل به دلیل قطعی برق تمامی اطلاعات را از دست دادیم به همین جهت پس از هر ایپاک یک بار مدل را ذخیره می‌کنیم تا در صورت رخ دادن اتفاقی مشابه بتوانیم آموزش را از همان قسمتی که دچار مشکل شده بود ادامه دهیم. نتایج اجرا در زیر آورده شده است:

\begin{table}[h]
  \centering
  \caption{نتایج اجرا بر روی مجموعه داده‌ی \متن‌لاتین{PeMSD7}}
  \begin{tabular}{|c|p{0.5\textwidth}|}
    \hline
    \متن‌لاتین{MAE} & مدل \\
    \hline
    \متن‌لاتین{RMSE} & متن لاتین{STGCN(Cheb)} \\
     $C_{O}$ & سایز کانال خروجی \\
    \hline
  \end{tabular}
  \label{tbl:distance}
\end{table}


جدول ناکامل

آموزش مدل بر روی مجموعه داده‌ی اسنپ با کلاستر لینوکس(پردازنده: \متن‌لاتین{Intel(R) C}، کارت گرافیک: \متن‌لاتین{NVIDIA Ge}) انجام شده است.تعداد ایپاک ها برابر ۵۰ قرار داده شد و در کل حدود ۲۲ ساعت زمان برد. نتایج اجرا در زیر آورده شده است


\begin{table}[h]
  \centering
  \caption{نتایج اجرا بر روی مجموعه داده‌ی \متن‌لاتین{PeMSD7}}
  \begin{tabular}{|c|p{0.5\textwidth}|}
    \hline
    \متن‌لاتین{MAE} & مدل \\
    \hline
    \متن‌لاتین{RMSE} & متن لاتین{STGCN(Cheb)} \\
     $C_{O}$ & سایز کانال خروجی \\
    \hline
  \end{tabular}
  \label{tbl:distance}
\end{table}

به طور مثال برای خیابان ولیعصر می‌خواهیم نمودار سرعت و سرعت پیش بینی شده را رسم می‌کنیم

\شروع{شکل}
  \درج‌تصویر[height=8cm]{./images/valiasr_osm.jpg}
  \تنظیم‌ازوسط
  \شرح{سیستم \متن‌لاتین{sharedstreet} در برابر سیستم اطلاعات جغرافیایی}
  \برچسب{fig:time-conv}
\پایان{شکل}

 داخل این \متن‌لاتین{osm way} شش \متن لاتین{SharedStreet} وجود دارد. برای \متن لاتین{SharedStreet} نشان داده شده در شکل اطلاعات را بر روی نمودار بریم.

\شروع{شکل}
  \درج‌تصویر[height=8cm]{./images/valiasr_shared.png}
  \تنظیم‌ازوسط
  \شرح{سیستم \متن‌لاتین{sharedstreet} در برابر سیستم اطلاعات جغرافیایی}
  \برچسب{fig:time-conv}
\پایان{شکل}

 \begin{lstlisting}
 SELECT * FROM predictions_by_time_15min_staging WHERE time_bucket=1826995 AND ref_bucket=10 AND ref='f6168cb90ac13a89112a6033b3047dad';
 \end{lstlisting}

 \زیرقسمت{خلاصه}
 ارزیابی مدل ارائه شده بر روی دو مجموعه داده \متن‌لاتین{PeMSD7} و اسنپ انجام و نتیجه گزارش شد. ارزیابی این مدل بر روی مجموعه داده‌ی اسنپ نیازمند مراحل بسیاری برای تبدیل \متن‌لاتین{GPS} ها به سرعت بود که همگی در این بخش توضیح داده شدند.
