\فصل{ارزیابی}

\پاراگراف{}
در نهایت به منظور ارزیابی پروژه از داده‌های ترافیکی واقعی \متن‌لاتین{PeMSD7} استفاده می‌کنیم که توسط اداره‌ی حمل و نقل ایالت کالیفرنیا\پانویس{https://pems.dot.ca.gov/} جمع آوری شده است.
این داده مربوط به ۲۲۸ ایستگاه می‌باشد که به صورت تصادفی از میان ۳۹ هزار ایستگاه‌ انتخاب شده‌اند و شامل ۱۲ هزار سطر است که نشان دهنده‌ی سرعت ترافیک در این ۲۲۸ ایستگاه با توالی زمانی پنج دقیقه می‌باشد.
داده‌ها در ماه‌های می تا جون سال ۲۰۱۲ جمع آوری شده‌اند و جهت پاک کردن داده‌های ترافیکی بی‌قاعده روزهای غیرکاری حذف شده‌اند.
برای مقایسه‌ی عملکرد این روش با روش‌های دیگر از معیارهای میانگین مطلق خطا\پانویس{MAE}، میانگین مطلق درصد خطا\پانویس{MAPE} و جذر میانگین مربعات خطا\پانویس{RMSE} استفاده می‌کنیم.
این روش را با روش‌های پایه‌ی شبکه‌ی عصبی پیش‌خور\پانویس{Feed-Forward Neural Network}، مدل خودهمبسته میانگین متحرک و حافظه‌ی کوتاه مدت ماندگار کاملا متصل\پانویس{Full-Connected LSTM} \مرجع{1409.3215} مقایسه خواهیم کرد.
