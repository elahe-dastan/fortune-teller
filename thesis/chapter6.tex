\فصل{ارزیابی}

در نهایت به منظور ارزیابی پروژه از دو مجموعه داده‌ی ترافیکی واقعی \متن‌لاتین{PeMSD7} و اسنپ! استفاده می‌کنیم.
برای مقایسه‌ی عملکرد این روش با روش‌های دیگر از معیارهای میانگین مطلق خطا\پانویس{MAE}، میانگین مطلق درصد خطا\پانویس{MAPE} و جذر میانگین مربعات خطا\پانویس{RMSE} استفاده می‌کنیم.
این روش را با روش‌های پایه‌ی شبکه‌ی عصبی پیش‌خور\پانویس{Feed-Forward Neural Network}، مدل خودهمبسته میانگین متحرک و حافظه‌ی کوتاه مدت ماندگار کاملا متصل\پانویس{Full-Connected LSTM} \مرجع{1409.3215} مقایسه خواهیم کرد.

\قسمت{\متن‌لاتین{PeMSD7}}

این مجموعه داده تمیز شده است و آماده‌ی دادن به مدل است. اجرای آموزش با کلاستر لینوکس(پردازنده: \متن‌لاتین{Intel(R) C}، کارت گرافیک: \متن‌لاتین{NVIDIA Ge}) انجام شده است.
تعداد ایپاک ها برابر ۵۰ قرار داده شد و در کل حدود ۳۸ ساعت زمان برد.
در حین آموزش مدل به دلیل قطعی برق تمامی اطلاعات را از دست دادیم به همین جهت پس از هر ایپاک یک بار مدل را ذخیره می‌کنیم تا در صورت رخ دادن اتفاقی مشابه بتوانیم آموزش را از همان قسمتی که دچار مشکل شده بود ادامه دهیم. نتایج اجرا در زیر آورده شده است:

\begin{table}[h]
  \centering
  \caption{نتایج اجرا بر روی مجموعه داده‌ی \متن‌لاتین{PeMSD7}}
  \begin{tabular}{|c|c|c|c|}
    \خط‌پر
    مدل & \متن‌لاتین{MAE} & \متن‌لاتین{MAPE (\%)} & \متن‌لاتین{RMSE}
    \خط‌پر
    \متن‌لاتین{ARIMA}
    \خط‌پر
  \end{tabular}
\end{table}

\قسمت{اسنپ!}

آموزش مدل بر روی مجموعه داده‌ی اسنپ انجام شده است. تعداد ایپاک ها برابر ۵۰ قرار داده شد و در کل حدود ۲۲ ساعت زمان برد. نتایج اجرا در زیر آورده شده است:

\begin{table}[h]
  \centering
  \caption{نتایج اجرا بر روی مجموعه داده‌ی اسنپ!}
  \begin{tabular}{|c|c|c|c|}
    \خط‌پر
    مدل & \متن‌لاتین{MAE} & \متن‌لاتین{MAPE (\%)} & \متن‌لاتین{RMSE}
    \خط‌پر
    \متن‌لاتین{ARIMA}
    \خط‌پر
  \end{tabular}
\end{table}

به طور مثال برای قسمتی از خیابان ولیعصر که در شکل؟؟؟؟؟؟؟؟؟؟ نشان داده شده‌است، نمودار سرعت و سرعت پیش بینی شده را رسم می‌کنیم

\شروع{شکل}
  \درج‌تصویر[height=8cm]{./images/valiasr_osm.jpg}
  \تنظیم‌ازوسط
  \شرح{سیستم \متن‌لاتین{sharedstreet} در برابر سیستم اطلاعات جغرافیایی}
  \برچسب{fig:time-conv}
\پایان{شکل}

 داخل این \متن‌لاتین{osm way} شش \متن‌لاتین{SharedStreet} وجود دارد. برای \متن‌لاتین{SharedStreet} نشان داده شده در شکل؟؟؟؟؟؟؟؟؟ اطلاعات را بر روی نمودار می‌بریم.

\شروع{شکل}
  \درج‌تصویر[height=8cm]{./images/valiasr_shared.png}
  \تنظیم‌ازوسط
  \شرح{سیستم \متن‌لاتین{sharedstreet} در برابر سیستم اطلاعات جغرافیایی}
  \برچسب{fig:time-conv}
\پایان{شکل}

از آن جایی که اکثر سفرهای این شرکت در بازه‌ی زمانی جهار و نیم تا هفت بعد از ظهر انجام می‌شود و تخمین زمان سفر در این بازه از اهمیت بالایی برخوردار است ما نیز نمودار زیر را در این بازه برای روز شنبه، شانزدهم بهمن ۱۴۰۰ رسم می‌کنیم


\شروع{شکل}
  \درج‌تصویر[height=8cm]{./images/p1.png}
  \تنظیم‌ازوسط
  \شرح{سیستم \متن‌لاتین{sharedstreet} در برابر سیستم اطلاعات جغرافیایی}
  \برچسب{fig:p1}
\پایان{شکل}


\شروع{شکل}
  \درج‌تصویر[height=8cm]{./images/p2.png}
  \تنظیم‌ازوسط
  \شرح{سیستم \متن‌لاتین{sharedstreet} در برابر سیستم اطلاعات جغرافیایی}
  \برچسب{fig:p2}
\پایان{شکل}


\شروع{شکل}
  \درج‌تصویر[height=8cm]{./images/p3.png}
  \تنظیم‌ازوسط
  \شرح{سیستم \متن‌لاتین{sharedstreet} در برابر سیستم اطلاعات جغرافیایی}
  \برچسب{fig:p3}
\پایان{شکل}


\شروع{شکل}
  \درج‌تصویر[height=8cm]{./images/p4.png}
  \تنظیم‌ازوسط
  \شرح{سیستم \متن‌لاتین{sharedstreet} در برابر سیستم اطلاعات جغرافیایی}
  \برچسب{fig:p4}
\پایان{شکل}



\قسمت{خلاصه}

ارزیابی مدل ارائه شده بر روی دو مجموعه داده \متن‌لاتین{PeMSD7} و اسنپ انجام و نتیجه گزارش شد. ارزیابی این مدل بر روی مجموعه داده‌ی اسنپ نیازمند مراحل بسیاری برای تبدیل \متن‌لاتین{GPS} ها به سرعت بود که همگی در این بخش توضیح داده شدند.
