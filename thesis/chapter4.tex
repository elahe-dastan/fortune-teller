\فصل{تعریف مساله}

برای درک بهتر مسئله همزمان با توضیح آن یک مثال را به طور موازی پیش می‌بریم.
هدف از این رساله بیان راه‌حلی برای پیش‌بینی سرعت ترافیک در برخی مختصات‌ها (مانند تعدادی میدان) در چند گام زمانی آینده است که
بدین جهت از سرعت ترافیک در این محل‌ها در گام‌های زمانی پیشین استفاده می‌کنیم. رابطه (\رجوع{eq:base}) این مساله سری زمانی را به زبان ریاضی توصیف می‌کند.

\begin{equation}
  \label{eq:base}
  \hat{v}_{t+1}, \ldots,  \hat{v}_{t+H} = \mathop{\mathrm{argmax}} \log \mathsf{P}({v}_{t+1}, \ldots,  v_{t+H} | v_{t-M+1} , \ldots,  v_{t})
\end{equation}

\begin{table}[h]
  \centering
  \caption{توضیح پارامترهای رابطه (\رجوع{eq:base})}
  \begin{tabular}{|c|p{0.5\textwidth}|}
    \hline
    $v_{t}$ & یک بردار به طول تعداد نقاطی که قصد داریم سرعت ترافیک را در آن‌ها پیش‌بینی کنیم که هر المان شامل سرعت ترافیک در یکی از مختصات‌های مورد نظر در زمان $t$ است. \\
    \hline
    $\hat{v}_{t+1}$ & سرعت پیش‌بینی شده در زمان $t+1$ \\
    \hline
    $H$ & تعداد گام های زمانی آینده که می‌خواهیم سرعت ترافیک را پیش بینی کنیم. \\
    \hline
    $M$ & تعداد گام های زمانی پیشین که برای پیش بینی استفاده می‌کنیم. \\
    \hline
  \end{tabular}
  \label{tbl:base}
\end{table}

برای مثال فرض کنیم می‌خواهیم سرعت ترافیک را در میدان فاطمی و میدان فلسطین (دو میدان معروف در تهران) پیش‌بینی کنیم.
$v_{t}$ یک بردار به طول دو خواهد بود که یک عضو آن سرعت ترافیک در میدان فاطمی در زمان $t$ مانند ساعت دو بعد از ظهر امروز و عضو دیگر آن شامل همین اطلاعات برای میدان فلسطین خواهد بود.
در این مثال $H$ را برابر یک و $M$ را برابر سه در نظر می‌گیریم. منظور از \رجوع{eq:base} این است که قصد داریم سرعت ترافیک در $H$ قدم زمانی بعدی را با دانستن $M$ قدم زمانی قبلی پیش‌بینی کنیم.
