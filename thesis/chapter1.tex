\فصل{مقدمه}

پیش‌بینی سرعت ترافیک برای کنترل و هدایت آن ضروری است.
به علت پیچیدگی و غیرخطی بودن سرعت ترافیک روش‌های قدیمی نمی‌توانند ترافیک را برای سفرهایی با زمان طولانی و متوسط به خوبی پیش‌بینی کنند.
پیش‌بینی دقیق و بی‌درنگ سرعت ترافیک برای افراد و سازمان‌های ارائه دهنده‌ی خدمات حمل و نقل و حتی دولت موضوع مهمی است چرا که حمل و نقل نقش مهمی در زندگی هر فرد ایفا می‌کند و یکی از اصلی‌ترین توانایی‌ها در سیستم ترابری هوشمند \پانویس{Intelligent Transportation System} به شمار می‌رود.
در این پروژه سعی داریم با استفاده از آموزش دادن شبکه‌های عصبی عمیق به پیش‌بینی سرعت ترافیک در برخی نقاط مشخص (مانند چهارراه‌ها و میدان‌ها) بپردازیم.
امروزه با استفاده از داده‌های عظیمی که در دسترس است و همچنین پیشرفت سخت‌افزار می‌توانیم شبکه‌های عمیقی که در گذشته قابل آموزش نبودند را آموزش دهیم و از توانایی بالای آن‌ها در پیش بینی مسائل پیچیده استفاده کنیم.

تخمین مدت زمان سفر مسئله‌ی پرکاربردی برای بسیاری از سازمان‌ها است. یکی از کلیدی ترین مراحل تخمین زمان سفر پیش‌بینی ترافیک است این موضوع به خصوص در سفرهای طولانی بسیار پراهمیت است. پیش از این کارهای متعددی در راستای حل این مسئله انجام شده است. از این کارها می‌توان به روش‌های آماری مانند رگراسیون خطی اشاره کرد که در سفرهای کوتاه مدت عملکرد خوبی دارند اما با طولانی‌تر شدن زمان سفر عملکردشان افت می‌کند. در سفرهای طولانی از روش‌هایی مانند مدل کردن پویا و روش‌های داده محور استفاده می‌شود امروزه با توجه به داده‌های زیادی که در اختیار داریم علاقه به سمت روش‌های داده محور مانند مدل‌های آماری کلاسیک و مدل‌های یادگیری ماشین  بیشتر است. در این گزارش ما نیز یک مدل یادگیری ماشین با رویکرد یادگیری عمیق را بحث خواهیم کرد و مورد ارزیابی قرار خواهیم داد. پیشتر از شبکه‌های عمیق مانند شبکه‌ی باور عمیق و کدکننده‌ی خودکار پشته‌ای برای پیش‌بینی ترافیک استفاده شده است اما این شبکه‌ها نمی‌توانند ویژگی‌های زمانی و مکانی را به طور مشترک از ورودی استخراج کنند.

مسئله بدین شکل است که می‌خواهیم برای تعدادی مختصات مشخص که به عنوان ورودی داده می‌شوند سرعت ترافیک را در گام‌های زمانی آینده پیش‌بینی کنیم و برای اینکار از سرعت ترافیک در این مختصات‌ها در گام‌های زمانی گذشته استفاده می‌کنیم. برای این پیش‌بینی از هر دو نوع ویژگی زمانی و مکانی بهره می‌بریم. با توجه به فاصله‌ی مختصات‌های مشخص شده از هم می‌توانیم گرافی رسم کنیم که گره‌های آن مختصات‌ها هستند و وزن یال‌ها با عکس فاصله‌ی مختصات‌ها از هم رابطه‌ی مستقیم دارد سپس برای آنکه بتوانیم از اطلاعات مکانی حداکثر بهره را ببریم مستقیما بر روی این گراف پیچش اعمال می‌کنیم. بدیهی است تغییر سرعت ترافیک در یک گره گراف باعث تغییر سرعت ترافیک در گره‌های مجاور می‌شود که میزان این تاثیر بر حسب فاصله‌ی گره‌ها از هم متفاوت است

در روش پیشنهادی سعی بر این است تا از هر دو ویژگی‌های زمانی و مکانی برای پیش‌بینی بهره بگیریم به همین جهت مدل، هر دو لایه‌های پیچشی زمانی و پیچشی مکانی را در خود دارد یکی از مسائل قابل توجه این است که در این روش از پیچش مستقیم بر روی گراف استفاده می‌شود که با استاندارد که بر تصاویر اعمال می‌شود متفاوت است. پیچیدگی زمانی پیچش بر روی گراف $O(n^{2})$ است برای سبک‌تر شدن احجم محاسبات می‌توانیم از تخمین مرتبه اول و یا تخمین چبیشف استفاده کنیم که در این گزارش ما از تخمین چبیشف کمک می‌گیریم

مسئله‌ای که با آن رو به رو هستیم جزي مسائل سری زمانی طبقه‌بندی می‌شود در چنین مسائلی شبکه‌های عصبی بازرخدادی پرکاربرد اند اما این شبکه‌ها دیر به تغییرات پاسخ می‌دهند به همین دلیل ما از ساختارهای کاملا پیچشی بر روی محور زمان استفاده می‌کنیم تا رفتار پویای سرعت ترافیک را دنبال کنیم.


از آن جایی که قصد داریم سرعت ترافیک را در چند گام زمانی (نه لزوما یک) آینده پیش‌بینی کنیم باید روشی برای آن اتخاض کنیم. برای پیش‌بینی سرعت ترافیک در یک گام زمانی آینده از تعدادی سرعت گذشته استفاده می‌کنیم، پنجره‌ای به اندازه‌ی تعداد این سرعت‌ها داریم، پس از پیش‌بینی سرعت برای یک گام آینده این پنجره را یک واحد به سمت جلو می‌بریم و در حقیقت از پیش‌بینی که برای یک گام زمانی بعد انجام داده بودیم برای پیش‌بینی سرعت در دو گام زمانی بعد استفاده می‌کنیم

برای ارزیابی روش پیشنهادی از دو مجموعه داده‌ی \متن‌لاتین{PeMSD7} و اسنپ! استفاده می‌کنیم. مجموعه‌داده‌ی \متن‌لاتین{PeMSD7} به صورت تمیز شده در اکوسیستم متن باز وجود دارد اما در مجموعه داده‌ی اسنپ! باید با استفاده از مختصات‌های جغرافیایی که توسط موقعیت‌یاب‌های‌جهانی تلفن‌های همراه رانندگان جمع‌آوری شده است سرعت خیابان‌ها را محاسبه کنیم و سپس از این سرعت‌ها برای آموزش مدل استفاده کنیم که نتایج مدل بر روی هر دو مجموعه داده رضایت بخش است.
