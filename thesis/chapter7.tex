\فصل{نتیجه‌گیری و کارهای آینده}
\قسمت{نتیجه‌گیری}
در این پروژه یک شبکه‌ی عصبی عمیق به نام شبکه‌ی پیچشی گرافی مکانی-زمانی را برای پیش‌بینی سرعت ترافیک معرفی کردیم که در آن با استفاده بلوک‌های پیچشی مکانی-زمانی، پیچش گرافی و پیچش زمانی دروازه‌ای ادغام شده بودند. ارزیابی‌ها نشان داد که این مدل عملکرد خوبی بر روی دو مجموعه داده‌ی واقعی داشت که نمایانگر پتانسیل‌های بزرگ آن در کاوش ساختارهای مکانی-زمانی از ورودی است.

\قسمت{کارهای آینده}
این مدل را می‌توان در سناریوهای پیش‌بینی توالی‌های ساختار یافته مکانی-زمانی عمومی‌تر، مانند تکامل شبکه‌های اجتماعی، و پیش‌بینی اولویت در سیستم‌های توصیه و غیره اعمال کرد.
