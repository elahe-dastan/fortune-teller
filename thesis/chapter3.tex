\فصل{کارهای مرتبط}

پیش‌بینی سرعت ترافیک بر مبنای طول سفر به دو دسته‌ی کلی کوتاه (۵ تا ۳۰ دقیقه) و متوسط-طولانی (بیشتر از ۳۰ دقیقه) تقسیم می‌گردد.
اکثر روش‌های آماری مانند رگراسیون خطی در سفرهای کوتاه مدت عملکرد خوبی دارند، اما به علت عدم قطعیت و پیچیدگی سرعت ترافیک این روش‌ها در سفرهای طولانی مدت دقت خوبی ندارند~\مرجع{1709.04875}.

مطالعات قبلی بر روی پیش‌بینی سرعت ترافیک در سفرهای بلند مدت را می‌توان به دو دسته‌ی مدل کردن پویا و روش‌های داده‌محور تقسیم کرد.
در روش‌های مدل کردن پویا از ابزار ریاضیاتی، مانند معادلات دیفرانسیل و دانش فیزیکی، برای فرموله کردن مسائل ترافیک استفاده می‌شود~\مرجع{Vlahogianni2015}.
در این روش برای رسیدن به حالت پایدار در پروسه‌ی شبیه‌سازی به سیستم‌های پیچیده و توانایی محاسباتی بسیار بالا نیاز است و فرضیه‌ها برای ساده‌سازی، دقت پیش‌بینی را کاهش می‌دهند.
به همین علت و همانطور که پیشتر بیان شد، به دلیل وجود داده‌ی زیادی که امروزه قابل دسترس است، علاقه به سمت روش‌های داده محور بیشتر است.
در روش‌های داده محور مدل‌های آماری کلاسیک و مدل‌های یادگیری ماشین دو نماینده‌ی اصلی هستند.

روش مدل خود همبسته میانگین متحرک\پانویس{Auto-Regressive Integrated Moving Average} و انواع آن یکی از روش های تجزیه و تحلیل سری زمانی است که مبتنی بر آمار کلاسیک\پانویس{Classical Statistics} است
و در طول زمان بسیار مورد بحث قرار گرفته است~\مرجع{Williams2003}،
اما این مدل محدود به فرضیات ثابتی درباره‌ی توالی‌های زمانی است و نمی‌تواند ارتباطات زمانی-مکانی را به حساب بیاورد. در نتیجه، این رویکردها قابلیت نمایش جریان ترافیک بسیار غیر خطی را محدود کرده‌اند.
اخیرا روش‌های یادگیری ماشین در پیش‌بینی سرعت ترافیک توانسته‌اند مدل‌های آماری کلاسیک را در وظایف پیش‌بینی سرعت ترافیک به چالش بکشند. با استفاده از مدل‌هایی مانند \متن‌لاتین{k} نزدیک‌ترین همسایه \پانویس{K nearest neighbors}، ماشین بردار پشتیبان \پانویس{support vector machine} و شبکه‌های عصبی \پانویس{neural networks} می‌توانیم به دقت بالاتر در پیش‌بینی و مدلسازی داده‌های پیچیده‌تر برسیم.

\قسمت{رویکردهای یادگیری عمیق}
امروزه روش‌های یادگیری عمیق با موفقیت بر روی مسايل ترافیکی اعمال شده‌اند و پیشرفت‌های زیادی در این زمینه صورت گرفته است، مانند شبکه‌ی باور عمیق\پانویس{Deep Belief Network}~\مرجع{YuhanJia2016} و کدکننده خودکار پشته‌ای\پانویس{Stacked Auto Encoder}~\مرجع{Chen_Song_Yamada_Shibasaki_2016}.
اما برای این شبکه‌های متراکم سخت است که بتوانند ویژگی‌های زمانی و مکانی را به طور مشترک از ورودی استخراج کنند.
همچنین در هنگام محدودیت و یا غیبت ویژگی‌های مکانی توانایی این شبکه‌ها تحت تاثیر قرار میگیرد.

برای استفاده‌ی کامل از ویژگی‌های مکانی، برخی محقق‌ها از شبکه‌ی عصبی پیچشی\پانویس{convolutional neural network} برای فهمیدن روابط مجاور در شبکه‌ی ترفیک استفاده می‌کنند،
همچنین شبکه‌ی عصبی بازرخدادی \پانویس{recurrent neural network} را بر روی محور زمان به کار می‌گیرند.
پژوهشگران ~\مرجع{Wu and Tan} با ترکیب شبکه‌ی حافظه‌ی کوتاه مدت ماندگار و شبکه‌ی عصبی پیچشی تک بعدی، یک معماری ترکیبی به نام \متن‌لاتین{CLTFP} در سطح ویژگی برای پیش بینی سرعت ترافیک در کوتاه مدت ارائه کردند. اگرچه \متن‌لاتین{CLTFP} یک استراتژی ساده را اتخاذ کرد، اما همچنان اولین تلاش را برای همسویی نظم مکانی و زمانی انجام داد.
پس از آن ~\مرجع{1506.04214} حافظه‌ی کوتاه مدت ماندگار پیچشی را پیشنهاد شد که یک حافظه‌ی کوتاه مدت ماندگار کاملا متصل \پانویس{fully connected long short term memory} با لایه‌های پیچشی تعبیه شده است. در این روش، عملیات پیچشی معمولی اعمال شده، مدل را فقط به پردازش ساختارهای شبکه‌ای (مانند تصاویر، ویدیوها) به جای دامنه‌های عمومی محدود می‌کند. در همین حال، شبکه‌های بازرخدادی برای یادگیری توالی نیاز به آموزش تکراری دارند که در هر مرحله تجمع خطا را گزارش می‌کند. علاوه بر این، شبکه‌های مبتنی بر شبکه‌های عصبی بازرخدادی (از جمله حافظه‌ی کوتاه مدت ماندگار) به طور گسترده‌ای شناخته شده‌اند که آموزش آن‌ها دشوار است و از نظر محاسباتی سنگین هستند.

چندین مطالعه‌ی یادگیری عمیق تازه‌تر نیز وجود دارد که انگیزه‌ی آن‌ها نیز پیچش گراف در وظایف مکانی-زمانی است. ~\مرجع{Seo} برای شناسایی ساختارهای مکانی مشترک و تغییرات دینامیکی در دنباله‌های ساختاریافته‌ی داده‌ها، شبکه‌های بازرخدادی پیچشی گراف \پانویس{graph convolutional reurrent network (GCRN)} را معرفی کرد. چالش اصلی این مطالعه تعیین ترکیب بهینه از شبکه‌های بازرخدادی و پیچش گراف تحت تنظیمات خاص است.

با استفاده از اصول بالا، پژوهشی~\مرجع{Li} با موفقیت از واحدهای بازگشتی دروازه‌ای \پانویس{gated recurrent unit} با پیچش گراف در پیش‌بینی ترافیک بلندمدت استفاده کرد. در مقابل این کارها، مدل پیشنهادی در این گزارش کاملا از ساختارهای کانولوشنی ساخته شده است. بلوک‌های پیچشی مکانی-زمانی به طور ویژه طراحی شده‌اند تا با اتصالات باقی‌مانده‌ای و استراتژی تنگنا\پانویس{bottleneck} که در داخل خود دارند داده‌های ساختاریافته را به صورت یکنواخت پردازش کنند. همچنین در این مدل از هسته‌های پیچش گراف کارآمدتری نیز استفاده می‌شود.


\زیرقسمت{خلاصه}
در این بخش روش‌های آماری که بر روی سفرهای کوتاه مدت عملکرد خوبی دارند و دو دسته‌ی روش‌های مدل کردن پویا و داده محور که در سفرهای بلند عملکرد بهتری دارند را بحث کردیم همچنین روش‌های یادگیری عمیق که برای پیش‌بینی سرعت ترافیک استفاده شده‌اند و نحوه‌ی عملکرد آن‌ها و ضعف‌هایشان را بیان کردیم که چگونه باعث پیشرفت در این زمینه شدند سپس تفاوت روش پیشنهادی و روش‌های قبلی را توضیح مختصری دادیم.
