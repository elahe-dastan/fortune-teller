\فصل{کارهای مرتبط}

پیش‌بینی سرعت ترافیک بر مبنای طول سفر به دو دسته‌ی کلی کوتاه (۵ تا ۳۰ دقیقه) و متوسط-طولانی (بیشتر از ۳۰ دقیقه) تقسیم می‌گردد.
اکثر روش‌های آماری مانند رگراسیون خطی در سفرهای کوتاه مدت عملکرد خوبی دارند، اما به علت عدم قطعیت و پیچیدگی سرعت ترافیک این روش‌ها در سفرهای طولانی مدت دقت خوبی ندارند~\مرجع{1709.04875}.

مطالعات قبلی بر روی پیش‌بینی سرعت ترافیک در سفرهای بلند مدت را می‌توان به دو دسته‌ی مدل کردن پویا و روش‌های داده‌محور تقسیم کرد.
در روش‌های مدل کردن پویا از ابزار ریاضیاتی، مانند معادلات دیفرانسیل و دانش فیزیکی، برای فرموله کردن مسائل ترافیک استفاده می‌شود~\مرجع{Vlahogianni2015}.
در این روش برای رسیدن به حالت پایدار در پروسه‌ی شبیه‌سازی به سیستم‌های پیچیده و توانایی محاسباتی بسیار بالا نیاز است و فرضیه‌ها برای ساده‌سازی، دقت پیش‌بینی را کاهش می‌دهند.
به همین علت و همانطور که پیشتر بیان شد، به دلیل وجود داده‌ی زیادی که امروزه قابل دسترس است، علاقه به سمت روش‌های داده محور بیشتر است.
در روش‌های داده محور مدل‌های آماری کلاسیک و مدل‌های یادگیری ماشین دو نماینده‌ی اصلی هستند.

روش مدل خود همبسته میانگین متحرک\پانویس{Auto-Regressive Integrated Moving Average} و انواع آن یکی از روش های تجزیه و تحلیل سری زمانی است که مبتنی بر آمار کلاسیک\پانویس{Classical Statistics} است
و در طول زمان بسیار مورد بحث قرار گرفته است~\مرجع{Williams2003}،
اما این مدل محدود به فرضیات ثابتی درباره‌ی توالی‌های زمانی است و نمی‌تواند ارتباطات زمانی-مکانی را به حساب بیاورد.
اخیرا روش‌های یادگیری ماشین در پیش‌بینی سرعت ترافیک توانسته‌اند مدل‌های آماری کلاسیک را به چالش بکشند.

امروزه روش‌های یادگیری عمیق با موفقیت بر روی مسايل ترافیکی اعمال شده‌اند و پیشرفت‌های زیادی در این زمینه صورت گرفته است،
مانند شبکه‌ی باور عمیق\پانویس{Deep Belief Network}~\مرجع{YuhanJia2016} و کدکننده خودکار پشته‌ای\پانویس{Stacked Auto Encoder}~\مرجع{Chen_Song_Yamada_Shibasaki_2016}.
اما برای این شبکه‌های متراکم سخت است که بتوانند ویژگی‌های زمانی و مکانی را به طور مشترک از ورودی استخراج کنند.
همچنین در هنگام محدودیت و یا غیبت ویژگی‌های مکانی توانایی این شبکه‌ها تحت تاثیر قرار میگیرد.
