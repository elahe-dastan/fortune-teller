%% -!TEX root = AUTthesis.tex
% در این فایل، عنوان پایان‌نامه، مشخصات خود، متن تقدیمی‌، ستایش، سپاس‌گزاری و چکیده پایان‌نامه را به فارسی، وارد کنید.
% توجه داشته باشید که جدول حاوی مشخصات پروژه/پایان‌نامه/رساله و همچنین، مشخصات داخل آن، به طور خودکار، درج می‌شود.
%%%%%%%%%%%%%%%%%%%%%%%%%%%%%%%%%%%%
% دانشکده، آموزشکده و یا پژوهشکده  خود را وارد کنید
\faculty{دانشکده مهندسی کامپیوتر}
% گرایش و گروه آموزشی خود را وارد کنید
\department{}
% عنوان پایان‌نامه را وارد کنید
\fatitle{طراحي و پياده سازي سيستم پيش‌بينی ترافيک بر پايه شبکه‌های عصبی}
% نام استاد(ان) راهنما را وارد کنید
\firstsupervisor{دکتر رضا صفابخش}
%\secondsupervisor{استاد راهنمای دوم}
% نام استاد(دان) مشاور را وارد کنید. چنانچه استاد مشاور ندارید، دستور پایین را غیرفعال کنید.
% \firstadvisor{نام کامل استاد مشاور}
%\secondadvisor{استاد مشاور دوم}
% نام نویسنده را وارد کنید
\name{الهه}
% نام خانوادگی نویسنده را وارد کنید
\surname{داستان}
%%%%%%%%%%%%%%%%%%%%%%%%%%%%%%%%%%
\thesisdate{زمستان ۱۴۰۰}

% چکیده پایان‌نامه را وارد کنید
\fa-abstract{}
پیش‌بینی سرعت ترافیک برای کنترل و هدایت آن ضروری است.
به علت پیچیدگی و غیرخطی بودن سرعت ترافیک روش‌های قدیمی نمی‌توانند ترافیک را برای سفرهایی با زمان طولانی و متوسط به خوبی پیش‌بینی کنند.
پیش‌بینی دقیق و بی‌درنگ سرعت ترافیک برای افراد و سازمان‌های ارائه دهنده‌ی خدمات حمل و نقل و حتی دولت موضوع مهمی است چرا که حمل و نقل نقش مهمی در زندگی هر فرد ایفا می‌کند و یکی از اصلی‌ترین توانایی‌ها در سیستم ترابری هوشمند \پانویس{Intelligent Transportation System} به شمار می‌رود.
در این پروژه سعی داریم با استفاده از آموزش دادن شبکه‌های عصبی عمیق به پیش‌بینی سرعت ترافیک در برخی نقاط مشخص (مانند چهارراه‌ها و میدان‌ها) بپردازیم.
امروزه با استفاده از داده‌های عظیمی که در دسترس است و همچنین پیشرفت سخت‌افزار می‌توانیم شبکه‌های عمیقی که در گذشته قابل آموزش نبودند را آموزش دهیم و از توانایی بالای آن‌ها در پیش بینی مسائل پیچیده استفاده کنیم.

% کلمات کلیدی پایان‌نامه را وارد کنید
\keywords{}



\AUTtitle
%%%%%%%%%%%%%%%%%%%%%%%%%%%%%%%%%%
