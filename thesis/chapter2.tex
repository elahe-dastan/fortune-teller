\فصل{مفاهیم پایه}

پیچش: اصطلاح پیچیدگی به ترکیب ریاضی دو تابع برای تولید تابع سوم اشاره دارد. این دو مجموعه از اطلاعات را ادغام می کند.

\begin{equation}
  \label{eq:base}
(f \text{*} g)(t) \triangleq \int_{-\infty}^\infty f(\tau)g(t - \tau)\tau'
\end{equation}


\begin{table}[h]
  \centering
  \caption{توضیح پارامترهای رابطه \رجوع{eq:time-conv}}
  \begin{tabular}{|c|p{0.5\textwidth}|}
    \hline
    $(f \text{*} g)(t)$ & توابعی که در حال پیچش هستند \\
    \hline
    $t$ & متغیر عددی حقیقی توابع \متن‌لاتین{f}  و \متن‌لاتین{g} \\
    \hline
    $g(\tau)$ & پیچش تابع $f(t)$ \\
    \hline
    $\tau'$ & مشتق اول تابع $g(\tau)$ \\
    \hline
  \end{tabular}
  \label{tbl:distance}
\end{table}

در مورد شبکه‌ی عصبی پیچشی، پیچش روی داده‌های ورودی با استفاده از یک فیلتر یا هسته انجام می‌شود (این اصطلاحات به جای یکدیگر استفاده می‌شوند) تا سپس یک نقشه ویژگی تولید شود. وقتی یک پیچش را با کشیدن فیلتر روی ورودی اجرا می کنیم. در هر مکان، یک ضرب ماتریس انجام می شود و نتیجه را بر روی نقشه ویژگی جمع می کند.

\شروع{شکل}
  \درج‌تصویر[height=8cm]{./images/convolution.png}
  \تنظیم‌ازوسط
  \شرح{پیچش}
  \برچسب{fig:time-conv}
\پایان{شکل}

\lr{spectral domain}

واحد بازگشتی دروازه‌ای: واحدهای بازگشتی دروازه‌ای یک مکانیزم دروازه‌ای در شبکه‌های عصبی بازرخدادی هستند. واحد بازگشتی دروازه‌ای مانند یک حافظه‌ی کوتاه مدت بلند با دروازه‌ی فراموشی است اما پارامترهای کم‌تری نسبت به  آن دارد، زیرا فاقد گیت خروجی است. عملکرد واحد بازگشتی دروازه‌ای در برخی از وظایف مدل‌سازی موسیقی چندصدایی، مدل‌سازی سیگنال گفتار و پردازش زبان طبیعی مشابه عملکرد حافظه‌ی کوتاه مدت ماندگار است. دو نسخه‌ی ساده و کامل از حافظه‌ی بازگشتی دروازه‌ای وجود دارد که روابط نسخه‌ی ساده‌ی آن در ادامه آورده شده است


\begin{equation}
  \label{eq:base}
  \begin{aligned}
  f_{t}& = \sigma _{g} ( W_{f} x_{t} + U_{f} h_{t-1} + b_{f} )\\
  {\hat {h}}_{t}& = \phi _{h}(W_{h}x_{t}+U_{h}(f_{t}\odot h_{t-1})+b_{h})\\
  h_{t}& = (1-f_{t})\odot h_{t-1}+f_{t}\odot {\hat {h}}_{t}
  \end{aligned}
\end{equation}


\begin{table}[h]
  \centering
  \caption{توضیح پارامترهای رابطه \رجوع{eq:time-conv}}
  \begin{tabular}{|c|p{0.5\textwidth}|}
    \hline
    $x_{t}$ & بردار ورودی \\
    \hline
    $h_{t}$ & بردار خروجی \\
    \hline
    ${\hat {h}}_{t}$ & بردار فعالسازی کاندید شده \\
    \hline
    $f_{t}$ & بردار فراموشی \\
    \hline
    $W, U ,b$ & ماتریس و بردار پارامترها \\
    \hline
  \end{tabular}
  \label{tbl:distance}
\end{table}
