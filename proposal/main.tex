\documentclass{article}

\usepackage[localise]{xepersian}

\settextfont{Vazir}

\begin{document}

\قسمت{تعریف مساله}
\پاراگراف{}
پیش‌بینی سرعت ترافیک برای کنترل و هدایت آن ضروری است.
به علت پیچیدگی و غیرخطی بودن جریان ترافیک روش‌های قدیمی نمی‌توانند ترافیک را برای سفرهایی با زمان طولانی و متوسط به خوبی پیش‌بینی کنند.
پیش‌بینی دقیق و بی‌درنگ سرعت ترافیک برای افراد و سازمان‌های ارائه دهنده‌ی خدمات حمل و نقل و حتی دولت موضوع مهمی است چرا که حمل و نقل نقش مهمی در زندگی هر فرد ایفا می‌کند و یکی از اصلی‌ترین توانایی‌ها در سیستم ترابری هوشمند \پانویس{intelligent transportation system} به شمار می‌رود.
در این پروژه سعی داریم با استفاده از آموزش دادن شبکه‌های عصبی عمیق به پیش‌بینی سرعت ترافیک در برخی نقاط مشخص (مانند چهارراه‌ها و میدان‌ها) بپردازیم.
امروزه با استفاده از داده‌های عظیمی که در دسترس است و همچنین پیشرفت سخت‌افزار می‌توانیم شبکه‌های عمیقی که در گذشته قابل آموزش نبودند را آموزش دهیم و از توانایی بالای آن‌ها در پیش بینی مسائل پیچیده استفاده کنیم.

\قسمت{مرور پیشینه}
\پاراگراف{}
پیش‌بینی سرعت ترافیک بر مبنای طول سفر به دو دسته‌ی کلی کوتاه (۵ تا ۳۰ دقیقه) ‌و متوسط-طولانی (بیشتر از ۳۰ دقیقه) تقسیم می‌گردد. اکثر روش‌های آماری مانند رگراسیون خطی در سفرهای کوتاه مدت عملکرد خوبی دارند، اما به علت عدم قطعیت و پیچیدگی جریان ترافیک این روش‌ها در سفرهای طولانی مدت دقت خوبی ندارند.

\end{document}
