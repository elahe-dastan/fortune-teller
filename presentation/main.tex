\documentclass{beamer}

\usetheme{metropolis}

\usepackage[style=numeric,sorting=ynt]{biblatex}
\usepackage[localise]{xepersian}

\setlatintextfont{Roboto}

\settextfont{Vazir}


% ---------------------------------------------------------------------------------
% Colors
% ---------------------------------------------------------------------------------
\definecolor{نارنجی}{rgb}{1.0, 0.31, 0.0}

% ---------------------------------------------------------------------------------
% Settings to force Beamer works with Xepersian and RTL typesetting
% ---------------------------------------------------------------------------------
% \raggedleft

% For right to left lists (itemize and enumerate)
\makeatletter
\newcommand{\لیست‌فارسی}{\raggedleft\rightskip\@totalleftmargin}
\makeatother
% Correct the bullet for RTL texts
\setbeamertemplate{itemize item}{\scriptsize\raise1.25pt%
  \hbox{\donotcoloroutermaths$\blacktriangleleft$}}

% To force beamer use numbering in captions
\setbeamertemplate{caption}[numbered]{}% Number float-like environments

\setbeamertemplate{footline}[frame number]
\setbeamertemplate{section in toc}[circle]
\setbeamercolor{block body}{bg=lightgray}
\setbeamercolor{headline}{bg=orange}
\setbeamersize{text margin left=1cm,text margin right=1cm}

\setbeamertemplate{headline}
{
  \begin{beamercolorbox}{section in head/foot}
    \vspace{2pt}\insertnavigation{\paperwidth}\vspace{2pt}
  \end{beamercolorbox}
}

% ---------------------------------------------------------------------------------
% To force beamer use numbering in captions
\setbeamertemplate{caption}[numbered]{}% Number float-like environments

\setbeamertemplate{footline}
{%
  \leavevmode%
  \hbox{%
    \begin{beamercolorbox}[wd=.333333\paperwidth,ht=2.25ex,dp=1ex,center]{author in head/foot}%
      \usebeamerfont{author in head/foot}\insertshortauthor%
    \end{beamercolorbox}%
    \begin{beamercolorbox}[wd=.333333\paperwidth,ht=2.25ex,dp=1ex,center]{title in head/foot}%
      \usebeamerfont{title in head/foot}\insertshorttitle%
    \end{beamercolorbox}%
    \begin{beamercolorbox}[wd=.333333\paperwidth,ht=2.25ex,dp=1ex,right]{date in head/foot}%
      \usebeamerfont{date in head/foot}\insertsection\hspace*{2em}
      \insertframenumber~ از \inserttotalframenumber{} \hspace*{2ex}%
    \end{beamercolorbox}
  }%
}
\setbeamertemplate{section in toc}[circle]
\setbeamertemplate{blocks}[rounded][shadow=false]
\setbeamercolor{block title}{bg=orange}
\setbeamercolor{block body}{bg=lightgray}
\setbeamersize{text margin left=1cm,text margin right=1cm}
\setbeamertemplate{frametitle continuation}{\insertcontinuationcount}

\author[الهه داستان]{%
  الهه داستان\hfill
  ۹۶۳۱۰۲۵\\
}
\title{پیش‌بینی ترافیک به وسیله شبکه‌های عصبی}
\subtitle{%
  پروژه کارشناسی\\
  دکتر صفابخش
}
\institute[]{دانشگاه صنعتی امیرکبیر}
\date{زمستان ۱۴۰۰}

\AtBeginSection[]
{
  \begin{frame}
    \frametitle{فهرست}
    \tableofcontents[currentsection]
  \end{frame}
}

\addbibresource{main.bib}

\begin{document}

\begin{persian}

\begin{frame}
  \titlepage{}
\end{frame}

\begin{frame}
  \frametitle{فهرست}
  \tableofcontents{}
\end{frame}

\section{مقدمه}

\begin{frame}
  \frametitle{هدف نهایی}
  \begin{block}{}
    مسافرت از یک نقطه مبدا تا یک نقطه مقصد چقدر طول می‌کشد؟
  \end{block}
\end{frame}

\begin{frame}
  \frametitle{هدف}
  \begin{columns}
    \column{.5\textwidth}
    \begin{block}{تعریف}
      پیش‌بینی سرعت در یک نقطه جغرافیایی مشخص در یک زمان مشخص
    \end{block}
    \pause
    \begin{block}{مثال}
      assume you want to go to the university and you want to know the speed of traffic at Theater Shahr on sunday at 9 am.
    \end{block}
    \pause
    \column{.45\textwidth}
    \includegraphics[height=0.5\textheight]{./img/dijkstra.png}
  \end{columns}
\end{frame}

\begin{frame}
  \frametitle{چه کسانی به آن احتیاج دارند؟}
  \begin{columns}
    \column{.25\textwidth}
    \includegraphics[height=0.5\textheight]{./img/google-maps.png}
    \textbf{Google Maps}\\
    \textit{Google}
    \column{.25\textwidth}
    \includegraphics[height=0.5\textheight]{./img/waze.png}
    \textbf{Waze}\\
    \textit{Google}
    \column{.25\textwidth}
    \includegraphics[height=0.5\textheight]{./img/balad.png}
    \textbf{Balad}\\
    \textit{نقشه پردازان ارتباطات نوین}
    \column{.25\textwidth}
    \includegraphics[height=0.5\textheight]{./img/neshan.png}
    \textbf{Neshan}\\
    \textit{پلتفرم نقشه نشان}
  \end{columns}
\end{frame}

\begin{frame}
  \frametitle{چالش‌ها\cite{2004.08555}}
  \شروع{فقرات}
  \لیست‌فارسی
  \فقره فاکتورهای خارجی
  \فقره داده ترافیکی به صورت زمانی-مکانی
  \شروع{فقرات}
  \لیست‌فارسی
  \فقره وابستگی‌های مکانی پیچیده
  \فقره وابستگی‌های زمانی پیچیده
  \پایان{فقرات}
  \پایان{فقرات}
  \centering
  \includegraphics[height=0.5\textheight]{./img/correlations.png}
\end{frame}

\begin{frame}
  \frametitle{تعریف مساله}
  \begin{equation}
    \label{eq:base}
    \hat{v}_{t+1}, \ldots,  \hat{v}_{t+H} = \mathop{\mathrm{argmax}} \log \mathsf{P}({v}_{t+1}, \ldots,  v_{t+H} | v_{t-M+1} , \ldots,  v_{t})
  \end{equation}
  \centering
  \includegraphics[height=.5\textheight]{img/V.png}
\end{frame}

\begin{frame}
  \frametitle{تعریف گرافی}
  \includegraphics[width=\textwidth]{img/convolution.png}
  \شروع{فقرات}
     \لیست‌فارسی
    \فقره با نزدیک شدن گره‌ها به یکدیگر ارتباط میان آن‌ها بیشتر می‌شود
    \فقره ابعاد ماتریس وزن‌ها: $nxn$
  \پایان{فقرات}
\end{frame}

\section{کارهای مرتبط}

\begin{frame}
  \frametitle{مقدمه\cite{1709.04875}}
  بعد از سال‌ها تحقیق پیوسته، تحقیقات در حوزه پیش‌بینی ترافیک پیشرفت زیادی داشته است.
  \شروع{فقرات}
  \لیست‌فارسی
  \فقره روش‌های سنتی: روش‌های آماری کلاسیک و روش‌های یادگیری ماشین
  \فقره روش‌های مبتنی بر یادگیری عمیق: مدل‌سازی وابستگی مکانی و زمانی
  \پایان{فقرات}
\end{frame}

\begin{frame}
  \frametitle{کارهای مرتبط}
  \شروع{فقرات}
  \لیست‌فارسی
  \فقره \متن‌سیاه{مدل خود هم‌بسته میانیگن متحرک}: تنها توالی‌های زمانی را در نظر می‌گیرند.
  \فقره \متن‌سیاه{شبکه باور عمیق}: ویژگی‌های زمانی و مکانی را به صور مشترک سخت از ورودی استخراج می‌کند.
  \فقره \متن‌سیاه{حافظه کوتاه مدت ماندگار}: سنگین از نظر محاسباتی
  \پایان{فقرات}
\end{frame}

\section{راه‌حل پیشنهادی}

\begin{frame}
  \شروع{فقرات}
  \لیست‌فارسی
  \فقره استفاده از هر دو نوع ویژگی زمانی و مکانی
  \فقره در نظر گرفتن مسیرها به صورت گرافی
  \فقره اعمال پیچش به صورت مستقیم بر روی گراف در دامنه طیفی
  \فقره اعمال پیچش بر روی محور زمان
  \پایان{فقرات}
\end{frame}

\begin{frame}
  \frametitle{ورودی راه حل}
  \includegraphics[width=\textwidth]{img/blackbox.png}
\end{frame}

\begin{frame}
  \frametitle{اطلاعات زمانی}
  \includegraphics[height=.5\textheight]{img/V.png}
\end{frame}

\begin{frame}
  \frametitle{اطلاعات مکانی}
  \includegraphics[height=.5\textheight]{img/matrix.png}
  \begin{equation}
    W_{i,j} = \left\{
      \begin{array}{ll}
        \exp(-\frac{d^{2}_{ij}}{\sigma^{2}}) & , i \neq j \quad and \quad \exp(-\frac{d^{2}_{ij}}{\sigma^{2}}) \geq \epsilon \\
        0 & , otherwise. \\
      \end{array}\right.
    \label{eq:distance}
  \end{equation}
\end{frame}

\begin{frame}
  \frametitle{معماری}
  \includegraphics[height=.8\textheight]{img/arch-1.png}
\end{frame}

\begin{frame}
  \frametitle{معماری}
  \includegraphics[height=.8\textheight]{img/arch-2.png}
\end{frame}

\begin{frame}
  \frametitle{معماری}
  \includegraphics[height=.8\textheight]{img/arch.png}
\end{frame}

\begin{frame}
  \frametitle{پیچش در دامنه طیفی}
  \begin{equation}
    \Theta *_{g} \chi = \Theta(L)\chi = \Theta(U \Lambda U^{T})\chi = U\Theta(\Lambda)U^{T}\chi
    \label{eq:convolution}
  \end{equation}
  \شروع{فقرات}
    \فقره پیچیدگی محاسباتی: $O(n^{2})$
  \پایان{فقرات}
\end{frame}

\begin{frame}
  \frametitle{چندجمله‌ای چبیشف}
  \begin{equation}
    \Theta \ast_{g} x = \Theta(L)x \approx \sum_{k=0}^{k-1} \Theta_{k} T_{k}(\widetilde{L})x
    \label{eq:approx-convolution}
  \end{equation}
  \شروع{فقرات}
     \لیست‌فارسی
    \فقره کاهش پیچیدگی
    \فقره محلی‌سازی
  \پایان{فقرات}
\end{frame}

\begin{frame}
  \frametitle{لایه‌ی پیچشی زمانی}
  \begin{equation}
    \Gamma *_{\tau} Y = P \odot \sigma (Q) \in R^{M-K_{t}+1 \times C_{O}}
    \label{eq:time-conv}
  \end{equation}
\end{frame}

\begin{frame}
  \frametitle{پیش‌بینی چند گام زمانی آینده}
  \includegraphics[width=\textwidth]{./img/recursive.png}
\end{frame}

\section{مجموعه‌ی داده‌ها و ارزیابی}

\begin{frame}
  \frametitle{مجموعه دادگان \متن‌لاتین{PeMSD7}}
  \شروع{فقرات}
  \لیست‌فارسی
  \فقره ۲۲۸ گره
  \فقره ماه‌های جون و می ۲۰۱۲
  \فقره برگرفته از سنسورها
  \پایان{فقرات}
\end{frame}

\begin{frame}
  \frametitle{مجموعه دادگان اسنپ}
  \شروع{فقرات}
  \لیست‌فارسی
  \فقره برگرفته از سامانه موقیعت‌یاب جهانی رانندگان
  \پایان{فقرات}
\end{frame}

\begin{frame}[allowframebreaks]
  \frametitle{مراجع}
  \begin{latin}
  \printbibliography[title=مراجع]
  \end{latin}
\end{frame}

\end{persian}

\end{document}
