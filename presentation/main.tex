\documentclass{beamer}

\usepackage[localise]{xepersian}

\usetheme[secheader]{Boadilla}
\usecolortheme{crane}

\author[الهه داستان]{%
  الهه داستان\hfill
  ۹۶۳۱۰۲۵\\
}
\title{پیش‌بینی ترافیک به وسیله شبکه‌های عصبی}
\subtitle{%
  پروژه کارشناسی\\
  دکتر صفابخش
}
\institute[AUT]{دانشگاه صنعتی امیرکبیر}
\date{زمستان ۱۴۰۰}

\AtBeginSection[]
{
  \begin{frame}
    \frametitle{فهرست}
    \tableofcontents[currentsection]
  \end{frame}
}

\usepackage[style=numeric,sorting=ynt]{biblatex}
\addbibresource{main.bib}

\setlatintextfont[
  Path=fonts/,
  UprightFont = *-Regular,
  BoldFont = *-Bold,
  ItalicFont = *-Italic
]{Neuton}

\settextfont[Path=fonts/]{BNazanin}

\begin{document}

\begin{persian}

\begin{frame}
  \titlepage{}
\end{frame}

\begin{frame}
  \frametitle{فهرست}
  \tableofcontents{}
\end{frame}

\section{مقدمه}

\begin{frame}
  \frametitle{هدف نهایی}
  \begin{block}{}
    مسافرت از یک نقطه مبدا تا یک نقطه مقصد چقدر طول می‌کشد؟
  \end{block}
\end{frame}

\begin{frame}
  \frametitle{هدف}
  \begin{columns}
    \column{.5\textwidth}
    \begin{block}{تعریف}
      پیش‌بینی سرعت در یک نقطه جغرافیایی مشخص در یک زمان مشخص
    \end{block}
    \pause
    \begin{block}{مثال}
      assume you want to go to the university and you want to know the speed of traffic at Theater Shahr on sunday at 9 am.
    \end{block}
    \pause
    \column{.45\textwidth}
    \includegraphics[height=0.5\textheight]{./img/dijkstra.png} % replace it with digistra
  \end{columns}
\end{frame}

\begin{frame}
  \frametitle{چه کسانی به آن احتیاج دارند؟}
  \begin{columns}
    \column{.25\textwidth}
    \includegraphics[height=0.5\textheight]{./img/google-maps.png}
    \textbf{Google Maps}\\
    \textit{Google}
    \column{.25\textwidth}
    \includegraphics[height=0.5\textheight]{./img/waze.png}
    \textbf{Waze}\\
    \textit{Google}
    \column{.25\textwidth}
    \includegraphics[height=0.5\textheight]{./img/balad.png}
    \textbf{Balad}\\
    \textit{Naghshe Pardazan Ertabatat Novin}
    \column{.25\textwidth}
    \includegraphics[height=0.5\textheight]{./img/neshan.png}
    \textbf{Neshan}\\
    \textit{Neshan Maps Platform}
  \end{columns}
\end{frame}

\begin{frame}
  \frametitle{چالش‌ها\footfullcite{2004.08555}}
  \begin{itemize}
    \item Traffic data is spatio-temporal
          \begin{itemize}
            \item Complex spatial dependencies
            \item Dynamic temporal dependencies
          \end{itemize}
    \item External factors
  \end{itemize}
  \centering
  \includegraphics[height=0.5\textheight]{./img/correlations.png}
\end{frame}

\begin{frame}
  \frametitle{تعریف مساله}
  \begin{equation}
    \label{eq:base}
    \hat{v}_{t+1}, \ldots,  \hat{v}_{t+H} = \mathop{\mathrm{argmax}} \log \mathsf{P}({v}_{t+1}, \ldots,  v_{t+H} | v_{t-M+1} , \ldots,  v_{t})
  \end{equation}
  \centering
  \includegraphics[height=.5\textheight]{img/base.png} % v.drawio
\end{frame}

\begin{frame}
  \frametitle{تعریف گرافی}
  \includegraphics[width=\textwidth]{img/convolution.png}
  \begin{itemize}
    \item با نزدیک شدن گره‌ها به یکدیگر ارتباط میان آن‌ها بیشتر می‌شود
    \item weight matrix dimension is $nxn$
  \end{itemize}
\end{frame}

\section{کارهای مرتبط}

\begin{frame}
  \frametitle{مقدمه\footfullcite{1709.04875}}
  After years of continuous researches, the research on traffic prediction has achieved great progresses.
  \begin{itemize}
    \item traditional methods: Classical Statistical Methods and Machine Learning Methods
    \item deep learning-based methods: Spatial Dependency Modeling and Temporal Dependency Modeling
  \end{itemize}
\end{frame}

\begin{frame}
  \frametitle{کارهای مرتبط}
  \begin{itemize}
  \فقره \متن‌سیاه{مدل خود هم‌بسته میانیگن متحرک}: تنها توالی‌های زمانی را در نظر می‌گیرند.
  \فقره \متن‌سیاه{شبکه باور عمیق}: ویژگی‌های زمانی و مکانی را به صور مشترک سخت از ورودی استخراج می‌کند.
  \فقره \متن‌سیاه{حافظه کوتاه مدت ماندگار}: سنگین از نظر محاسباتی
  \end{itemize}
\end{frame}

\section{راه‌حل پیشنهادی}

\begin{frame}
  \شروع{فقرات}
  \فقره استفاده از هر دو نوع ویژگی زمانی و مکانی
  \فقره در نظر گرفتن مسیرها به صورت گرافی
  \فقره اعمال پیچش به صورت مستقیم بر روی گراف در دامنه طیفی
  \فقره اعمال پیچش بر روی محور زمان
  \پایان{فقرات}
\end{frame}

\begin{frame}
  \frametitle{ورودی راه حل}
  % blackbox
\end{frame}
\begin{frame}
  اطلاعات زمانی
  % v.drawio
\end{frame}
\begin{frame}
  اطلاعات مکانی
  % matrix.drawio
  رابطه اسلاید distance
\end{frame}

\begin{frame}
  \frametitle{Distance}
  \begin{equation}
    W_{i,j} = \left\{
      \begin{array}{ll}
        \exp(-\frac{d^{2}_{ij}}{\sigma^{2}}) & , i \neq j \quad and \quad \exp(-\frac{d^{2}_{ij}}{\sigma^{2}}) \geq \epsilon \\
        0 & , otherwise. \\
      \end{array}\right.
    \label{eq:distance}
  \end{equation}
\end{frame}

\begin{frame}
  \frametitle{Architecture}
  % arch.drawio
\end{frame}

\begin{frame}
  % پیچش در دامنه طیفی
  % رابطه ۴-۲
\end{frame}

\begin{frame}
  \frametitle{Spectral Graph Convolution}
  \begin{equation}
    \Theta *_{g} \chi = \Theta(L)\chi = \Theta(U \Lambda U^{T})\chi = U\Theta(\Lambda)U^{T}\chi
    \label{eq:convolution}
  \end{equation}
  \begin{itemize}
    \item computational complexity is $O(n^{2})$
  \end{itemize}
\end{frame}

\begin{frame}
  \frametitle{Chebyshev Polynomials}
  \begin{equation}
    \Theta \ast_{g} x = \Theta(L)x \approx \sum_{k=0}^{k-1} \Theta_{k} T_{k}(\widetilde{L})x
    \label{eq:approx-convolution}
  \end{equation}
  \begin{itemize}
    \item reduce the complexity
    \item localization
  \end{itemize}
\end{frame}

\begin{frame}
  لایه‌ی پیچشی زمانی
  رابطه ۴-۵
\end{frame}

\begin{frame}
  پیش‌بینی چند گام زمانی آینده
\end{frame}

\section{Datasets and Evaluation}

\begin{frame}
  \frametitle{PeMSD7}
  \begin{itemize}
    \item 228 Nodes
    \item May and June 2012
    \item From sensors
  \end{itemize}
\end{frame}
\begin{frame}
  \frametitle{Snapp}
  \begin{itemize}
    \item GPS Probes
  \end{itemize}
\end{frame}

\begin{frame}[allowframebreaks]
  \frametitle{مراچع}
  \begin{latin}
  \printbibliography[title=مراجع]
  \end{latin}
\end{frame}

\end{persian}

\end{document}
